\documentclass{apuntes}

\title{Pr\'{a}ctica 4 - ARQO}
\date{Diciembre 2013}
\author{V\'{i}ctor de Juan Sanz - Guillermo Juli\'{a}n Moreno}

\begin{document}
\maketitle
\newpage

\pagestyle{plain}

\section*{Ejercicio 0}

Características de las CPU de los ordenadores del laboratorio:

\begin{description}
\item[Número de cores físicos] 2.
\item[Número de cores virtuales] 4.
\item[Hyperthreading] Sí.
\item[Frecuencia] $1.2\,\mathrm{GHz}$.
\end{description}

\section*{Ejercicio 1}

\paragraph{¿Cómo se comporta OpenMP cuando declaramos una variable privada?} OpenMP declara instancias privadas de la variable por cada hilo que genera, de tal forma que cada hilo tiene su propia variable a cuyo valor no acceden el resto de hilos.

\paragraph{¿Qué ocurre con el valor de una variable privada al comenzar a ejecutarse la región paralela?} Se inicializa al valor que tenía antes de comenzar la región paralela.

\paragraph{¿Qué ocurre con el valor de una variable privada al finalizar la región paralela?} La variable privada mantiene el valor que tenía antes de ejecutarse la región paralela.

\paragraph{¿Ocurre lo mismo con las variables públicas?} No, las variables públicas son accesibles por todos los hilos y por lo tanto cualquier modificación que hagan los hilos se mantendrá al acabar la región paralela.

\section*{Ejercicio 2}

El resultado es correcto en la versión en serie y en la versión 2 de la paralela. La versión 1 paralela no funciona porque al final no suma los resultados parciales de cada uno de los hilos. Además, en la versión 2 puede haber errores de redondeo en los números de coma flotante debido a que suma en un orden distinto al de la versión serie.


% \easyimg{Ej2_Tiempos.png}{Tiempos de ejecución}{imgEj2Tiempos}
% \easyimg{Ej2_Acel.png}{Aceleración}{imgEj2Accel}

Para obtener mejores mediciones, hemos multiplicado por 100 los tamaños de matrices a ejecutar, de tal forma que las pequeñas variaciones que pueda introducir el SO no afecten tanto a la medición. También ejecutamos varias veces el bucle y obtenemos la media del tiempo de ejecución para evitar la aparición de medidas anómalas.

\subsection*{Preguntas}

\paragraph{En términos del tamaño de los vectores, ¿compensa siempre lanzar hilos para realizar el trabajo en paralelo, o hay casos en los que no?}

\end{document}